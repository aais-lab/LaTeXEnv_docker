% ドキュメントの余白
\documentclass[12pt,a4paper]{jreport}
\setlength{\textwidth}{38zw}
\setlength{\oddsidemargin}{1cm}
\setlength{\marginparwidth}{0pt}
\setlength{\topmargin}{0.2cm}
\renewcommand{\baselinestretch}{1.3}
\setlength{\textheight}{37\baselineskip}
\addtolength{\textheight}{\topskip}

% 参考文献
\renewcommand{\bibname}{参考文献}
\renewcommand{\theenumi}{\roman{enumi}}
\bibliographystyle{junsrt}
\usepackage{makeidx}
\pagestyle{headings}
\usepackage{graphicx}
\pagenumbering{roman}
\usepackage{url}
\usepackage{amsmath}

% 枠関連の設定
\usepackage{fancybox}
\usepackage{framed}
\newenvironment{longbox}{%
  \def\FrameCommand{\fboxsep=\FrameSep \fbox}%
  \MakeFramed {\FrameRestore}}%
 {\endMakeFramed}

\makeindex

\begin{document}


%
% 表紙
%
\begin{titlepage}
\begin{center}
{\large 令和X年度修士学位論文\\}
 \vspace{3cm}
 {\Huge タイトル\\
\vspace{5mm}に関する研究\\}
\vspace{4cm}
{\large%
  千葉工業大学~~先進工学研究科~~知能メディア工学専攻\\
\vspace{3mm}22S30xx~~名前\\
\vspace{3cm}指導教員~~今野~~将~~教授\\
\vspace{1cm}20XX年3月\\
}
\end{center}
\end{titlepage}

%%%%%%%%%%%%%%%%%%%%%%%%%%和文要旨%%%%%%%%%%%%%%%%%%%%%%%%%%%%%%
\begin{titlepage}
\begin{longbox}
\begin{center}
 {\Huge タイトル}\\
 {\large%
   \vspace{3mm}%
   22S30xx~~名前\\
 }
\end{center}
{\bf 概要:}%




\end{longbox}
\end{titlepage}
%%%%%%%%%%%%%%%%%%%%%%%%%%%%%%%%%%%%%%%%%%%%%%%%%%%%%%%%%%%%%%%%




% 目次
\newpage
\tableofcontents

% ページ番号設定
\newpage
\setcounter{page}{0}\pagenumbering{arabic}
%%%%%%%%%%%%%%%%%%%%%% 本文 %%%%%%%%%%%%%%%%%%%%%%
% 各章毎にファイルを分割して\inputで読み込ませると管理が楽

%%%%%%%%%%%%%%%%%%%%%%%%%%%%%%%%%%%%%%%%%%%%%%%%%%%%%%%%%%%%%%%%

\chapter*{謝辞}
\addcontentsline{toc}{chapter}{謝辞}
\nocite{*}
%%%%%%%%%%%%%%%%%%%%%%%%%%% 謝辞 %%%%%%%%%%%%%%%%%%%%%%%%%%%

謝辞を書く

%%%%%%%%%%%%%%%%%%%%%%%%%%%%%%%%%%%%%%%%%%%%%%%%%%%%%%%%%%%%%

% 参考文献
\nocite{*}
\newpage
\pagestyle{plain}
\addcontentsline{toc}{chapter}{参考文献}
%参考文献ファイル(Bibファイル)を使って参考文献リストを作る場合
\bibliography{reference}
%参考文献ファイルの名前は上記の場合 reference.bibになる

%%%%%%%%%%%%%%%%%%%%%%%%%%% 付録 %%%%%%%%%%%%%%%%%%%%%%%%%%%
% \appendix
% 本文同様,chapterなどを使用して書く

%図の一覧を出力する場合以下のコメントを外す
%\listoffigures

%表の一覧を出力する場合以下のコメントを外す
%\listoftables
%%%%%%%%%%%%%%%%%%%%%%%%%%%%%%%%%%%%%%%%%%%%%%%%%%%%%%%%%%%%%


\end{document}
